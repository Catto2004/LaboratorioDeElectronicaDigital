% --- Librerías para el texto ---
\documentclass{replab}
\usepackage{lipsum}
\usepackage{graphicx}
\usepackage{geometry}
\usepackage{tikz}
\usepackage{lastpage}
\usepackage[spanish,english]{babel}
\usepackage{changepage}
\usepackage{lettrine}
\usepackage{ragged2e}
\usepackage[T1]{fontenc}
\usepackage{enumitem}
\usepackage{palatino}
\usepackage[font=small,labelfont=bf,labelsep=space]{caption}
\usepackage{upgreek}
\usepackage{cancel}
\usepackage{amsmath}
\usepackage{mathdots}
\usepackage{xcolor}
\usepackage{mathrsfs}
\usepackage{stackrel}
\usepackage{type1ec}
\usepackage{type1cm}
\usepackage{subfig}
\usepackage{wrapfig}
\usepackage{wasysym}
\usepackage{enumerate}
\usepackage{adjustbox}
\usepackage[dvipsnames,svgnames,table]{xcolor}
\usepackage{fancyhdr}
\usepackage{amssymb}
\usepackage{amsfonts}
\usepackage{amsbsy}
\usepackage{amsthm}
\usepackage{algorithmic}
\usepackage{float}
\usepackage{multirow}
\usepackage{textcomp}
\usepackage{tcolorbox}
\tcbuselibrary{listingsutf8}
\tcbuselibrary{theorems}
\usepackage{hyperref}
\usepackage{url}
\usepackage{listings}
\usepackage{xcolor}

\definecolor{codeblue}{rgb}{0.0, 0.0, 0.6}
\definecolor{codegray}{rgb}{0.5, 0.5, 0.5}
\definecolor{codegreen}{rgb}{0.0, 0.5, 0.0}
\definecolor{backcolor}{rgb}{0.95, 0.95, 1.0}

\lstdefinestyle{systemverilog}{
    language=Verilog,
    backgroundcolor=\color{backcolor},
    basicstyle=\ttfamily\small,
    keywordstyle=\color{codeblue}\bfseries,
    commentstyle=\color{codegreen}\itshape,
    stringstyle=\color{codegray},
    numbers=left,
    numberstyle=\tiny\color{codegray},
    stepnumber=1,
    numbersep=10pt,
    tabsize=4,
    showstringspaces=false,
    breaklines=true,
    frame=single,
    captionpos=b
}

% --- Información del documento ---
\title{Práctica Nº4}
\author{\textbf{Tutor:} Ing. J. O. Sánchez -R
\thanks{Aunque esta plantilla es obra del autor, el texto utilizado para mostrar la disposición del documento es una adaptación del artículo en \autocite{bhandari-2022}.}}

\date{\today}
\subtitle={Flip -Flops}
\email={\href{mailto:icarous@utp.edu.co}{\color{principaluno}\texttt{icarous@utp.edu.co}}}
\subject={Cátedra de Laboratorio Fundamentos de Electrónica}

\setlength{\columnsep}{14pt}

% --- Archivo de bibliografía ---
\addbibresource{repbib.bib}

% --- Inicio del documento ---
\begin{document}
	
\pagestyle{fancy}
\unspacedoperators
	
% --- Título ---
\twocolumn[
 \begin{center}
  \maketitle
  {\begin{tcolorbox}[colframe=white, colback=principaldos, arc=8pt]
				
\begin{onecolabstract}
Este documento contiene informeación referente el análisis de circuitos con basculadores (Flip-Flops), para evaluar el comportamiento de las configuraciones tipo [SR, JK, D, T] como las bases para las memorias de datos, teniendo cuenta las configuraciones PIPO, PISO, SISO, SIPO.

\medskip
\noindent\textit{\textbf{Palabras clave:}} Flip-flops, tablas de estados comportamiento, SR, JK, D, T, configuraciones PIPO, PISO, SISO, SIPO.
\end{onecolabstract}

\tcblower
 \selectlanguage{english}
\begin{onecolabstract}
This document contains information regarding the analysis of circuits with flip-flops,
to evaluate the behavior of configurations such as [SR, JK, D, T] as the basis for
data memories, taking into account the PIPO, PISO, SISO, and SIPO configurations.

\medskip
\noindent\textit{\textbf{Keywords:}} Flip-flops, state tables, SR, JK, D, T, configurations, PIPO, PISO, SISO, SIPO
\end{onecolabstract}
   \end{tcolorbox}}
  \smallskip
\end{center}
]

\selectlanguage{spanish}
	
% --- Cuerpo del reporte ---
\section{Introducción}

Los Flip-Flops o basculadores son dispositivos electrónicos digitales fundamentales utilizados para el almacenamiento de información binaria dentro de sistemas computacionales. Estos dispositivos permiten conservar estados lógicos añtos y bajos dependiendo los flancos entre valores [\textbf{0} = maxterm = bajo, \textbf{1} = maxterm = alto] y modificarlo en función de señales de control, por lo que son esenciales en el diseño de memorias, contadores, registros y máquinas de estado. El estudio de los diferentes tipos de Flip-Flops, su funcionamiento y sus aplicaciones permite comprender la base del procesamiento secuencial en electrónica digital.

En este informe se analizan los principales tipos de Flip-Flops, sus características, tablas de estados lógicos y posibles implementaciones en circuitos digitales. Además; se incluyen esquemas que ilustran la estructura interna y el comportamiento de cada uno de ellos.

Este documento está organizado para el lector de la siguiente manera: \textbf{Sección 1} - Introducción, \textbf{Sección 2} - Conceptos, \textbf{Sección 3} - Análisis y resultados, \textbf{Sección 4} - Conclusiones, \textbf{Sección 5} - Referencias.


\vspace{0.3cm}

\section{Conceptos}

\subsection{Flip-Flops}
Un Flip-Flop o basculador es un dispositivo electrónicos con dos estados [A-Estable, Bi-Estable] que se utiliza como la unidad de memoria más básica en la electrónica digital, capaz de almacenar un solo bit de información binaria.

\begin{figure}[H]
    \centering
    \includegraphics[width=0.8\linewidth]{Imágenes/Práctica/f1.png}
    \caption{Flip-Flops}
    \label{fig:etiqueta}
\end{figure}

\subsubsection{Tipos}

\subsubsection{Tablas de Estados de comportamiento}
Qué son las tablas de estados reset, set, retención, indeterminado?
\begin{figure}[H]
    \centering
    \includegraphics[width=0.8\linewidth]{Imágenes/Práctica/f2.png}
    \caption{Tipos Flip-Flops}
    \label{fig:etiqueta}
\end{figure}

\subsubsection{FF SR}
¿Qué es FF SR?.

\subsubsection{FF SR}
¿Qué es FF JK?.

\subsubsection{FF D}
¿Qué es FF D?.

\subsubsection{FF T}
¿Qué es FF JK?.

\subsection{Configuraciones}
¿Qué son las configuraciones y para que sirven en memorias y contadores?

\subsubsection{Clúster PIPO}
Poner definición breve... Con gráficos.

\subsubsection{Clúster PISO}
Poner definición breve... Con gráficos.

\subsubsection{Clúster SISO}
Poner definición breve... Con gráficos.

\subsubsection{Clúster SIPO}
Poner definición breve... Con gráficos.

\newpage
\section{Análisis y Resultados}

Aquí empieza a colocar las imágenes del desarrollo de los circuitos sin carreta, el código utilizado y las simulaciones con los debidos argumentos breves.

Con éste código se insertan imágenes.

\begin{equation}
 \label{EQ:Ec1}
  \begin{split}
   m_{\rho} = \frac{-b \pm \sqrt{b^2 - 4ac}}{2a}
  \end{split}
\end{equation}

Código para anexar en látex para anexar el verilog.

\begin{lstlisting}[style=systemverilog, caption=HDL - NOT, basicstyle=\footnotesize\ttfamily, numbers=none]
//Compuerta NOT
module not_gate(
  input Xa,
  output Y
);
  assign Y = ~Xa;
endmodule
\end{lstlisting}

\textbf{\textcolor{red}{Nota}}: Incluya para cada circuito de las configuraciones (PIPO, PISO, SISO, SIPO) las tablas de comportamiento de estados obtenidas.

Código para tablas de estado lógico
\subsubsection*{Tabla Estados Lógicos}

\begin{center}
\begin{tabular}{c c | c}
S & R & Q(next) \\
\hline
0 & 0 & Q(prev) \\
0 & 1 & 0 \\
1 & 0 & 1 \\
1 & 1 & \text{Indeterminado} \\
\end{tabular}
\end{center}

O éste otro modelo si lo prefiere
\begin{table}[H]
\centering
\begin{tblr}{
    colspec = {|c|c|c|c|c|},
    hlines
}
\textbf{A} & \textbf{B} & \textbf{$C_{in}$} & \textbf{Suma} & \textbf{$C_{out}$} \\
0011 & 0101 & 0 & 1000 & 0 \\
1111 & 0001 & 0 & 0000 & 1 \\
\end{tblr}
\caption{Resultados de suma}
\end{table}

\section{Conclusiones}
Las concluisones van numeradas, no tiradas, fuera de eso las conclusiones van en (VERBOS EN FUTURO).

\begin{enumerate}
  \item Se evidenció qué...
  \item Validamos qué...
  \item El experimento demostró qué...
  \item Se colocan al menos [5] conclusiones con análisis de ingeniero.
\end{enumerate}

\begin{thebibliography}{99}
\bibitem{b1} Electricidad y electrónica 2010 - Autor: Agustín Rela - ISBN: C1229ACE - MinEducación Argentina - INET Saavedra - Buenos Aires Argentina - Pág 33 - 55.
\bibitem{b2} Fundamentos de circuitos eléctricos - Autor: Charles K.Alexander / Matthew N. O. Sadiku - ISBN: 978-607-15-0948-2 - McGraw-Hill México - Pág 199.
\bibitem{b3} Máquinas eléctricas - Autor: Stephen J.Chapman - ISBN: 978-607-15-0724-2 - McGraw-Hill México - Pág 205 - 219.
\end{thebibliography}

\vspace{9cm}
\begin{figure}[H]
 \centering
   \includegraphics[width=0.2\linewidth]{Imágenes/Institucional/Logo-UTP.png}
\end{figure}

\end{document}
